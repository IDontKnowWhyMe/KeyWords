\documentclass{article}
\usepackage[margin=1.5in]{geometry}
\usepackage{listings}
\usepackage{xcolor}

\lstset{
  basicstyle=\ttfamily,
  columns=fullflexible,
  breaklines=true,
  postbreak=\mbox{\textcolor{red}{$\hookrightarrow$}\space},
}

\title{Uživatelská dokumentace pro klicova$\_$slova}
\author{Lukáš Štupák}
\date{\today}

\begin{document}

\maketitle

\section{Použití}
\begin{lstlisting}[language=bash]
perl klicova_slova.pl [moznosti]
\end{lstlisting}

Extrahuje a seřadí klíčová slova z textu poskytnutého na standardním vstupu.

\section{Možnosti}
\begin{itemize}
  \item \texttt{--stopwords DIR}: Cesta ke slovníku stop slov (výchozí: stopwords).
  \item \texttt{--lang LANG}: Jazyk pro stemování a stop slova (výchozí: cs).
  \item \texttt{--num\_keywords NUM}: Počet výstupních klíčových slov (výchozí: 10).
  \item \texttt{--output\_format FMT}: Formát výstupu (text, json, csv) (výchozí: text).
  \item \texttt{--help}: Zobrazí tuto nápovědu.
\end{itemize}

\section{Popis}
Skript \texttt{script.pl} extrahuje a seřadí klíčová slova z vstupního textu. Načte stop slova ze souboru a použije algoritmus TextRank k určení důležitosti slov. Výsledná klíčová slova se vypíšou na standardní výstup ve formátu specifikovaném možností \texttt{--output\_format}.

\section{Příklady použití}
\begin{enumerate}
  \item Extrahovat a vypsat 10 klíčových slov z vstupního textu:
  \begin{lstlisting}[language=bash]
  perl klicova_slova.pl
  \end{lstlisting}

  \item Extrahovat a vypsat 5 klíčových slov z vstupního textu ve formátu JSON:
  \begin{lstlisting}[language=bash]
  perl klicova_slova.pl --num_keywords 5 --output_format json
  \end{lstlisting}

  \item Extrahovat a vypsat 15 klíčových slov z vstupního textu do CSV souboru:
  \begin{lstlisting}[language=bash]
  perl klicova_slova.pl --num_keywords 15 --output_format csv > keywords.csv
  \end{lstlisting}
\end{enumerate}

\section{Poznámky}
\begin{itemize}
  \item Skript předpokládá, že balíčky Lingua::Stem::Snowball a Getopt::Long jsou nainstalovány. Pokud nejsou, můžete je nainstalovat pomocí následujících příkazů:
  \begin{lstlisting}[language=bash]
  cpan Lingua::Stem::Snowball
  cpan Getopt::Long
  \end{lstlisting}

  \item Skript očekává, že stop slova pro zvolený jazyk jsou k dispozici ve souboru s názvem \texttt{<lang>.txt} v adresáři \texttt{stopwords}, kde \texttt{<lang>} je kód jazyka (např. "cs" pro češtinu). Pokud takový soubor neexistuje, skript bude pokračovat bez použití stop slov.

  \item Vstupní text je očekáván na standardním vstupu. Můžete jej zadat ručně nebo upravit zdrojový kód skriptu tak, aby četl z jiného zdroje.
\end{itemize}

\end{document}
